\chapter{Concluzie}\label{conclusions}

\section{Avantaje și obstacole}\label{avantaje_obstacole_intro}

Am gândit \AppName{} în jurul ideii democratizării informațiilor financiare. Utilizarea bonurilor fiscale ca date de intrare pentru aplicație aduce avantajul de a nu depinde de modul în care a fost făcută achiziția. Majoritatea utilizatorilor au mai multe carduri, dar folosesc și bani lichizi pentru unele achiziții. Soluția propusă adună toate aceste tranzacții într-un singur loc.

Un alt avantaj este flexibilitatea datelor. Soluția propusă oferă o vizualizare rapidă a tranzacțiilor în aplicație, asemenea altor soluții, dar oferă și exportul datelor, pentru ca acestea să fie folosite pentru analize detaliate în programe cum ar fi \emph{Excel}.

Principalul obstacol al acestei aplicații este înțelegerea automată a bonurilor fiscale. Am împărțit această sarcină în două probleme: \emph{OCR} și extragerea informațiilor relevante din textul oferit de OCR. Deși problema recunoașterii caracterelor este considerată rezolvată în condiții ideale, aceasta pune în continuare probleme sub condiții imperfecte. În cadrul acestei aplicații, condițiile pentru OCR sunt date de hârtie de proastă calitate, posibil mototolită, cerneală ștearsă și poze care pot fi de proastă calitate și pun probleme soluțiilor existente. Extragerea informațiilor din textul bonurilor fiscale este și ea o problemă dificilă deoarece bonurile nu respectă un format standard, iar rezultatul OCR-ului poate să nu fie perfect. 
