\chapter*{Concluzii}\label{conclusions}
\addcontentsline{toc}{chapter}{Concluzii}  

Am gândit \AppName{} în jurul ideii democratizării informațiilor financiare. Utilizarea bonurilor fiscale ca date de intrare pentru aplicație aduce avantajul de a nu depinde de modul în care a fost făcută achiziția. Majoritatea utilizatorilor au mai multe carduri, dar folosesc și bani lichizi pentru unele achiziții. Soluția propusă adună toate aceste tranzacții într-un singur loc.

Un alt avantaj este flexibilitatea datelor. Soluția propusă oferă o vizualizare rapidă a tranzacțiilor în aplicație, asemenea altor soluții, dar oferă și exportul datelor, pentru ca acestea să fie folosite pentru analize detaliate în programe cum ar fi \emph{Excel}.

Principalul obstacol întâmpinat în dezvoltarea aplicației a fost înțelegerea automată a bonurilor fiscale. Abordarea aleasă are două mari dezavantaje. Primul este acela că performanța ei este limitată de performanța modulului \emph{OCR} ales. Al doilea dezavantaj este lipsa flexibilității și mai ales imposibilitatea ca extragerea informațiilor să se îmbunătățească fără a face modificări în codul aplicației.

Funcționalitatea de export o găsesc foarte utilă pentru urmărirea cheltuielilor personale. Câteva întrebări la care se poate răspunde având la îndemână nu doar lista tranzacțiilor, ci și produsele de pe fiecare chitanță sunt: \emph{Ce sumă am cheltuit în această lună pe apă, pâine etc.?}, \emph{Cu cât sunt mai scumpe produsele la supermarket-ul X comparativ cu Y?}. De asemenea, aceste date sunt și o înregistrare a evoluției prețurilor de-a lungul timpului. Adunarea acestor date manual este o sarcină laborioasă, pe care puțini o fac. \AppName este o unealtă care să faciliteze această sarcină.

Din punct de vedere al implementării, limbajul \emph{Kotlin} aduce o experiență de dezvoltare mult mai plăcută comparativ cu \emph{Java 7}. Librăriile din cadrul \emph{Android Architecture Components} încearcă să ofere soluții moderne pentru structurarea și implementarea aplicațiilor. Totuși, menținerea compatibilității cu versiunile vechi ale \emph{API}-urilor \emph{Android} limitează gradul de dezvoltare al acestora. Un \emph{bug} rezolvat recent care evidențiază această problemă este păstrarea subscribției unui fragment la un \emph{viewModel} după ce acest fragment și-a încheiat ciclul de viață, ceea ce conduce la \emph{memory leaks}. Privind retrospectiv, poate că \emph{Flutter}, un \emph{framework} mult mai nou, dezvoltat în limbajul \emph{Dart} ar fi fost o alegere mai bună pentru implementarea acestei aplicații.

O experiență plăcută am avut lucrând cu \emph{RxJava}. Această librărie are o complexitate ridicată și necesită un efort considerabil din partea programatorului pentru a o înțelege. Avantajul primit în schimb este ușurința cu care se poate executa cod asincron. De asemenea, abordarea funcțională a acestei librării face testarea mai ușoară și reduce din posibilele \emph{bug-uri}.

Dezvoltarea ulterioară a acestei aplicații vizează publicarea acesteia și promovarea pe medii online, cum ar fi \emph{Reddit} sau \emph{ProductHunt}. De asemenea, această aplicație este în mod \emph{open source} și poate aduna contribuții de la mai mulți programatori, care pot îmbunătăți metoda de extragere a informațiilor din chitanțe.
