\chapter{Introducere}\label{introducere}

Această lucrare prezintă \AppNameB, o aplicație mobilă de scanare a bonurilor fiscale și vizualizare a cheltuielilor, disponibilă pentru platforma \emph{Android}.

\section{Motivație}

Urmărirea și analizarea cheltuielilor este o sarcină importantă pentru alcătuirea unui buget personal și pentru o organizare mai bună a activităților financiare. Popularizarea în ultimii ani a plăților electronice și cu cardul a facilitat apariția a tot mai multe astfel de servicii automate. Majoritatea băncilor oferă astăzi o aplicație mobilă cu funcționalitatea de a urmări și clasifica tranzacțiile clienților.

Cea mai mare problemă pe care aceste servicii o întâmpină este lipsa unor date mai bogate, fără de care valoarea pe care o pot aduce este limitată. Într-adevăr, soluțiile existente valorifică un set limitat de date disponibile tranzacțiilor, printre care numele comerciantului și suma totală. Fără a avea acces la conținutul tranzacției, serviciile existente nu pot oferi o listă comprehensivă cu toate achizițiile utilizatorului.

O altă problemă a serviciilor oferite de bănci pentru urmărirea cheltuielilor este disponibilitatea datelor. Utilizatorii acces limitat la datele ce le aparțin. Aceste date sunt disponibile fie doar în aplicația băncii, fie pot fi exportate în formate ce nu pot fi valorificate mai departe, cum ar fi documente PDF. În acest caz, o întrebare simplă, cum ar fi \textit{Cât am cheltuit în această lună pe pâine?} devine greu de răspuns.

Având în vedere dezavantajele soluțiilor curente, am dezvoltat \AppName, o aplicație care să ofere o mai mare vizibilitate asupra tranzacțiilor financiare. Aceasta permite scanarea bonurilor fiscale, înțelegerea automată a acestora, prezentarea grafică și stocarea acestora într-o bază de date locală și exportul acestora în cloud, de unde pot fi descărcate pentru o analiză mai amănunțită utilizând uneltele utilizatorului.

\section{Obiective}

În procesul de dezvoltare a aplicației \AppName{} mi-am propus:
\begin{enumerate}
  \item 
  \textbf{Dezvoltarea și îmbunătățirea algoritmului de extragere de informații}: Extragerea informațiilor structurate din imagini este un proces complex, ce nu poate avea o soluție standard, care să funcționeze în orice caz. În plus, lipsa unei metode formale de evaluare a sarcinii de înțelegere a conținutului chitanțelor din imagini acceptată la scară largă îngreunează dezvoltarea de noi metode. Acest proiect aplică un set de metode euristice, cu scopul de înțelege o gamă cât mai largă de bonuri fiscale populare în România și de a necesita un efort minim din partea utilizatorului;
  \item
  \textbf{Valorificarea confidențialității utilizatorului}: Datele financiare ale utilizatorilor pot fi fructificate de către agenții de publicitate din mediul online și de aceea utilizatorii pot fi reticenți în a folosi o aplicație care are acces la acestea. Am gândit această aplicație astfe încât comunicarea cu un server să se facă doar voluntar și complet anonim, astfel încât informațiile despre achiziții să nu poată fi legate de un anumit utilizator;
  \item
  \textbf{Implementarea unor standarde înalte ale structurii codului}: Calitatea codului și organizarea acestuia are un puternic impact asupra succesului unui proiect software pe termen mediu și lung. În cadrul acestui proiect mi-am propus explorarea bunelor practici în dezvoltarea aplicațiilor \emph{Android} și construirea unei arhitecturi care să faciliteze testarea și decuplarea sistemului și care să fie ușor de înțeles și implementat.
\end{enumerate}

\section{Lucrări asemănătoare}\label{related_intro}

Analizarea și procesarea chitanțelor financiare pe baza imaginilor obținute folosind camera foto a telefoanelor a stârnit un interes moderat în comunitatea științifică și tehnică. 

\begin{itemize}
  \item
    (Janssen et al. 2012) \cite{Receipts2Go}
  \item
    (Raoui-Outach et al. 2017) \cite{DL_receipt_understanding}
  \item
    (Ullah et al. 2018) \cite{Heuristics_extract}
  \item
    {[}https://www.neat.com/{]}
\end{itemize}

\section{Structura lucrării}\label{structura_intro}

